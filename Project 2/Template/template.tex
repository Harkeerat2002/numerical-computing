\documentclass[unicode,11pt,a4paper,oneside,numbers=endperiod,openany]{scrartcl}

\input{assignment.sty}


\begin{document}


\setassignment
\setduedate{Wednesday, 25 October 2023, 11:59 PM}

\serieheader{Numerical Computing}{2023}{Student: FULL NAME}{Discussed with: FULL NAME}{Solution for Project 2}{}
\newline

\assignmentpolicy

\newpage

%%%%%%%%%%%%%%%%%%%%%%%%%%%%%%%%%%%%%%%%%%%%%%%%%%
\section{The assignment}
%%%%%%%%%%%%%%%%%%%%%%%%%%%%%%%%%%%%%%%%%%%%%%%%%%

\subsection{Implement various graph partitioning algorithms [50 points]}
Summarize your results in table \ref{table:bisection}.


\begin{table}[h]
\caption{Bisection results}
\centering
\begin{tabular}{l|r|r|r|r} \hline\hline 
Mesh             &  Coordinate           & Metis 5.0.2  & Spectral & Inertial  \\ \hline
grid5rec(12,100)&   12                   &             &          &           \\             
grid5rec(100,12)&   12                   &             &          &           \\ 
grid5recRotate(100,12,-45)&              &             &          &           \\ 
gridt(50)        &                        &             &          &           \\ 
grid9(40)        &                        &             &          &           \\ 
Smallmesh            &                        &             &          &           \\ 
Tapir            &                        &             &          &           \\ 
Eppstein            &                        &             &          &           \\  
\hline \hline
\end{tabular}
\label{table:bisection}
\end{table}

%----

\subsection{Recursively bisecting meshes [20 points]}

Summarize your results in table \ref{table:Rec_bisection}.

\begin{table}[h]
\caption{Edge-cut results for recursive bi-partitioning.}
\centering
\begin{tabular}{l|r|r|r|r} \hline\hline 
 Case            &  Spectral             &  Metis 5.1.0    & Coordinate & Inertial  \\ \hline
 mesh3e1         &                       &                 &            &           \\             
 bodyy4        &                       &                 &            &           \\ 
 de-2010            &                       &                 &            &           \\ 
 biplane-9          &                       &                 &            &           \\ 
 L-9           &                       &                 &            &           \\  \hline \hline
\end{tabular}
\label{table:Rec_bisection}
\end{table}

%----

\subsection{Comparing recursive bisection to direct $k$-way partitioning [15 points]}

Summarize your results in table \ref{table:Compare_Metis}.

\begin{table}[h]
\caption{Comparing the number of cut edges for recursive bisection and direct multiway partitioning in Metis 5.1.0.}
\centering
\begin{tabular}{l|r|r} \hline\hline 
Partitions       &   Helicopter           & Skirt  \\ \hline
 16 - recursive bisection             &                       &     \\   
 16-way direct partition             &                       &             \\           
 32 - recursive bisection                &                       &             \\
 32-way direct partition              &                       &             \\  \hline \hline
\end{tabular}              
\label{table:Compare_Metis}
\end{table}



\end{document}
